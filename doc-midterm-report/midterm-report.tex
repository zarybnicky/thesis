% Created 2019-01-26 Sat 00:27
% Intended LaTeX compiler: pdflatex
\documentclass[11pt,a4paper]{scrartcl}
\usepackage[utf8]{inputenc}
\usepackage[T1]{fontenc}
\usepackage{graphicx}
\usepackage{grffile}
\usepackage{longtable}
\usepackage{wrapfig}
\usepackage{rotating}
\usepackage[normalem]{ulem}
\usepackage{amsmath}
\usepackage{textcomp}
\usepackage{amssymb}
\usepackage{capt-of}
\usepackage{hyperref}
\author{Jakub Zárybnický}
\date{\today}
\title{A Haskell Platform for Creating Progressive Web Applications\\\medskip
\large A midterm progress report}
\hypersetup{
 pdfauthor={Jakub Zárybnický},
 pdftitle={A Haskell Platform for Creating Progressive Web Applications},
 pdfkeywords={},
 pdfsubject={},
 pdfcreator={Emacs 26.1 (Org mode 9.1.9)}, 
 pdflang={English}}
\begin{document}

\maketitle
\tableofcontents


\section{Introduction}
\label{sec:org4022981}
In this report, I will introduce the project to the reader and describe its goals
and their current status as of the end of January 2019. In particular, I will
focus on what my original task looked like and how I deviated from it, and how I
expect to correct these flaws going forward.

I will also describe the work I've performed so far and propose a plan for the
remaining part.

\section{Project}
\label{sec:org440b3be}
My original idea for the project was to build a set of libraries and patterns
that would ease the development of full-stack Haskell web applications, both
their client and server parts. The work would fill a niche in the language's
ecosystem that is in demand, and it would build a basis for further development
of tools and libraries.

When I introduced the project to my adviser, I focused in particular on
Progressive Web Applications, a rather popular topic in general web
development. I think it was a mistake, in retrospect, and that I should have
emphasized the 'full-stack' aspect of my original idea, which would expose me to
a greater challenge of creating a 'pattern language' to describe different
shapes of web applications, which I find more interesting. I have spent some
time and effort on this topic, and I hope to incorporate it into the final
result of my thesis.

The stated goals of the project at its beginning were:

\begin{enumerate}
\item Study the current state of the Haskell ecosystem for creating web
applications.
\item Find suitable libraries as a starting point for creating Progressive Web
Applications (PWAs), i.e., web applications that can offer the user
functionality such as working offline or push notifications.
\item Implement a framework for PWAs. Focus, in particular, on the implementation
of components for offline storage, push notifications, and also support
tools.
\item Create a set of example PWAs using the created framework.
\item Compare the created framework with existing (e.g. JavaScript) frameworks for
PWAs.
\item Summarize the obtained results and discuss the future work.
\end{enumerate}

The goal for this first midterm checkpoint was to complete the first two points
at a minimum.

\section{Work performed}
\label{sec:org81c739e}
I would say that I have completed the first two points and made some sizable
progress on the third one. While my focus has been on other projects for most of
the past half-year, I have continued researching relevant topics for most of the
time and I have also made some significant progress on the applications from
which I take inspiration for this project.

My work since the start of the project included:
\begin{itemize}
\item A cross-sectional analysis of the available frameworks for many languages, to
get a taste for the conveniences that developers from other languages are used
to.
\item An analysis of the web-development tools and libraries available in the
Haskell ecosystem.
\item I have done further work on the applications upon which I am basing the
framework. I have also created one and a half other new applications with a
different architecture from the first ones (the second one is only half done).
\item I have extracted several patterns from the above applications into a
repository of 'snippets'. These snippets will later become the starting point
for the framework itself.
\item I have researched many topics related to current web development trends
(JAM stack, Serverless, Web Components, CRDTs, DevOps topics, best practices
in authentication and authorization, and others).
\end{itemize}

Visible artifacts of my work are in two places:
\begin{itemize}
\item A Git repository at \url{https://github.com/zarybnicky/thesis}
\item An almost empty blog at \url{https://zarybnicky.com}
\item One in-progress article at
\url{https://zarybnicky.com/drafts/elements-of-pwa-framework/}
\end{itemize}

Some artifacts aren't ready to be published yet, but I hope to remedy that
during the last week of January or first week of February. These artifacts are:
\begin{itemize}
\item A finished and published version of the "Elements of a Web framework" article
\item An article on all the Haskell tools and libraries that are available, and a
decision about which ones are suitable.
\item More publicly available 'snippets' extracted from the existing applications.
\end{itemize}

I would also like to publish the source code for at least one of the several
applications that are the inspiration for this work, but I do not expect to get
permission from all stakeholders of the projects.

Regarding the last point of my 'work accomplished' list, the research on current
web development trends - I do not expect to go through my entire backlog of
notes and bookmarks anytime soon, but I hope to start working through it and
writing up my notes into the form of articles. I have started a habit of writing
long-form text daily, so I hope these articles will take less than two months to
write - unlike the last one.

\section{Problems encountered}
\label{sec:org4fffc61}
My main problem during the period from September to December is a lack of
available time. In the time before the start of the school year - and also now
in January - I've had a much lighter load, and so I was able to work on less
urgent projects which include this thesis. The demands for my time are
unfortunately irregular, but I now have a rather clear idea about the next steps
for the project, which brings me to the next problem--

Until this week, I was rather uncertain about the real scope of the
project. That meant, among other things, that there wasn't any visible forward
movement in the project, only the constant broadening of scope. I've had to
clarify the goals while writing this report and preparing the presentation
slides, which means that I will be able to schedule my work better.

My last problem was more of a personal one - my productivity neared zero during
October and November, and everything I did took much longer than necessary,
which meant that any time I might have had for my thesis went to other, more
urgent work.

\section{Further plans and schedule}
\label{sec:org5e9a410}
Going over every aspect of the project in the last week while writing the
presentation and getting the Git repository into shape meant that I had to
confront how large my envisioned scope for this project became. I had to prune
out all tasks not directly related to the real goals of the thesis.

Out of several long thinking and writing sessions came several versions of a
plan. These plans are tiered, and each includes everything from the previous
plan. I call these plans 'realistic' - what I know I can have ready at the end
of the project regardless of outside influences; 'optimistic' - what I think is
doable but I don't expect to have fully finished by the project deadline; and
'highly optimistic' - my idea of an ideal outcome, I expect to start working on
its goals and perhaps have some ready, but not entirely and definitely not as a
polished product.

\subsection{Realistic}
\label{sec:org804976a}
This plan is only composed of the original objectives of this thesis. The result
is a set of libraries and scripts for writing a browser application that
fulfills all the requirements of a Progressive Web Application as defined on
Google's checklist.

This checklist notably includes:
\begin{itemize}
\item Pages are responsive on tablets \& mobile devices
\item All app URLs load while offline
\item Metadata provided for Add to Home screen
\item Page transitions don't feel like they block on the network
\item Each page has a URL
\item Pages use the History API
\item Site uses cache-first networking
\item Site appropriately informs the user when they're offline
\item Push notifications (consists of several related requirements)
\end{itemize}

To achieve this, I need to create:
\begin{itemize}
\item A full-featured browser routing library. While there are some existing
implementations, they are either incomplete or long abandoned.
\item A wrapper around ServiceWorkers, or a template to simplify project creation.
\item A push notifications library.
\item A library or a script that will render HTML 'shells' of all pages on the site,
for fast first load.
\end{itemize}

Not required by the checklist, but would improve the quality of my work:
\begin{itemize}
\item A template of an application, with predefined internal architecture, that uses
all of the above libraries
\item A utility library for querying and caching data from an API, be it HTTP, WebSocket,
GraphQL, or others.
\end{itemize}

An application using these tools and libraries would consist of a server written
independently, and of a browser application. Implementing the communication
between them is left to the developer, as is implementing many common
conveniences usually provided by a framework.

These goals also include fully documenting the code written, as well as testing
and benchmarking it to remove the most obvious bottlenecks.

\subsection{Optimistic}
\label{sec:org48d561d}
In addition to the previous plan, this plan includes a communication channel
between the server and client, with a data storage component in the client and a
synchronization protocol for use in both real-time applications and fully
offline-capable applications.

The result will then also include:
\begin{itemize}
\item a library to use in code shared between the server and client - a way to
define the shape of the transport channel (and its API for non-Haskell
applications)
\item a server library, to allow user code to implement the specified protocol
\item a client library with a storage component for entities and pending requests
\end{itemize}

There are multiple sub-goals in this goal. There is no frontend cache library
yet, so a library that implements transparent caching would be sufficient
here. Designing a synchronization protocol that would enable applications to
stay offline for prolonged periods of time and synchronizing whenever they come
online would be a great achievement, and perhaps a topic for a bachelor thesis
on its own\ldots{}

An application using these libraries would consist of a server and a client
sharing code that contains the definition of their communication channel. I do
not yet know how much these libraries would affect the shape/architecture of the
server, but the client library would form the core of the client - with the
libraries from the previous plan forming the shell.

In an ideal case, designing the synchronization protocol would include a proof
of correctness as well. I have seen several ways to achieve this, the most
recent one being a project called hs-to-coq.

And, like above, this plan also includes documentation, tests, and benchmarks of
the created code.

\subsection{Highly optimistic}
\label{sec:org06cde1c}
The nature of this plan's goals is a bit different. Whereas I solve small,
well-defined problems in the previous ones, this plan's goal is to design the
types to describe all possible shapes of web applications. This is inspired by
my research into the 'JAM stack' and related architectures, where most of the
client code is executed during build-time or even on demand, and only user
interface code remains to be executed in the browser.

The goal here is to design an abstraction that covers, at a minimum, all of
these cases:
\begin{itemize}
\item Client only. An application that doesn't need to communicate with a server,
like a web presentation, or a blog, a set of pre-compiled HTML+JS files.
\item Server only. Either just an API, or a plain HTML website with no JavaScript.
\item Server and client, with the client rendered during run-time by the server.
\item Server and client, with the client rendered during build-time and served
separately (e.g. via an S3 bucket).
\item Server and client, with the client re-rendered on demand, whenever the data that
it shows changes. This is the shape of a 'JAM stack' application.
\end{itemize}

This also includes projects with multiple clients or servers, which puts even
more pressure on the supporting tools.

This is more of exploratory work, so I don't expect to see results
immediately. However, I have projects of several of the proposed shapes (a
client-only app, a run-time rendered client, and a 'JAM stack' application) in
which I could try out any attempts at a design, to see how useful or easy-to-use
they are.

\section{Conclusion}
\label{sec:orgc177080}
I am, of course, aiming for the most optimistic plan, as that's that presents
both the biggest challenge and offers the greatest benefits. Just to be safe
though, I will need to start with the 'realistic' plan, so that I avoid
sprinting at the last moment at a time when I should be preparing for my final
exam.

I still feel that my expectations are set too high and that I will need to lower
them a lot, especially regarding the second and third plan, but I hope that my
'realistic' plan is realistic enough for real-life.
\end{document}