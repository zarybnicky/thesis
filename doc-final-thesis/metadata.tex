%%% Local Variables:
%%% mode: latex
%%% TeX-master: "projekt.org"
%%% End:

\projectinfo{
  project={BP},
  year={2019},
  date=\today,
  title.cs={Nástroj pro tvorbu Progressive Web Applications v Haskellu},
  title.en={A Haskell Platform for Creating \mbox{Progressive}\\Web~Applications},
  author.name={Jakub},
  author.surname={Zárybnický},
  department={UITS},
  faculty={FIT},
  supervisor.name={Ondřej},
  supervisor.surname={Lengál},
  supervisor.title.p={Ing.},
  supervisor.title.a={Ph.D.},
  keywords.cs={Haskell, GHCJS, Web Platform, Progressive Web Application, open source, návrh knihoven},
  keywords.en={Haskell, GHCJS, Web Platform, Progressive Web Application, open source, library design},
  abstract.cs={

    Tato práce se snaží usnadnit vývoj webových aplikací psaných v programovacím
    jazyce Haskell vytvořením sady komponent, které zatím chybí v jeho
    ekosystému knihoven, se zaměřením na komponenty nutné pro tvorbu Progressive
    Web Applications, t.j.~aplikace, které používají nové technologie jako
    např.~Service Workers.

    Tato práce porovnává, které komponenty se očekávají od webových platforem a
    které jsou dostupné pro Haskell; popisuje implementaci tří komponent
    (knihovny pro routování, úložiště a Service Workery); a implementuje tři
    aplikace, které demonstrují použití těchto komponent.

  },
  abstract.en={

    This work attempts to ease developing browser applications in the Haskell
    programming language by creating a set of components that its library
    ecosystem so far lacks, especially focusing on the components required for
    development of Progressive Web Applications, i.e.~applications that use new
    technologies like Service Workers.

    The thesis compares which components are commonly expected from a web
    framework and which are available in Haskell; describes the implementation
    of three such components (router, storage, and Service Worker libraries);
    and implements three applications that demonstrate use of these components.

  },
  declaration={

    Hereby I declare that this bachelor's thesis was prepared as an
    original author’s work under the supervision of Ing.~Ondřej Lengál,~Ph.D.
    All the relevant information sources used during preparation of this
    thesis are properly cited and included in the list of references.

  },
  extendedabstract={

    Webové aplikace se už dnes svou velikostí a komplexitou blíží nativním a je
    možné je psát v mnoha jazycích. V této práci se zaměřuji na tvorbu aplikací
    pro prohlížeče v jazyce Haskell, který ale disponuje omezenou nabídkou
    dostupných knihoven v této oblasti. Zaměřuji se obzvlášť na tzv.~Progressive
    Web Applications, což jsou aplikace, které používají sadu relativně nových
    technologií jako např.~Service Workers, aby poskytly funkce, které byly
    dříve dostupné jen v nativních aplikacích. Takové aplikace jsou schopné
    např.~pracovat offline, nebo používat tzv.~push notifikace, upozornění
    generovaná serverem.

    V této práci nejdříve porovnávám funkce dostupné v nejrozšířenějších
    webových platformách, tj.~funkce, které očekávají i vývojáři, kteří Haskell
    používají nebo chtějí používat, s funkcemi aktuálně dostupnými v jazyce
    Haskell. Na základě této analýzy pak implementuji tři komponenty, které jsou
    podle mého uvážení nejvíce potřeba pro vývoj Progressive Web Applications a
    které se navzájem doplňují: knihovny pro routování, úložiště a Service
    Workers (jejich generování a interakci s nimi).

    Knihovna pro routování umožňuje definici mapování URL na GUI komponenty a
    přechody mezi URL a stránkami převážně pomocí programování na úrovni typů
    (\emph{type-level programming}). Knihovna také obsahuje prototyp komponenty
    pro statické generování stránek (\emph{static site generation}). Knihovna
    pro úložiště obsahuje jednoduché úložiště typu klíč-hodnota (\emph{key-value
    storage}), s podporou offline úložiště a možností rozšíření pro přímý
    přístup k databázi při statickém generování stránek. Komponenta pro podporu
    Service Workers má dvě části, generování Service Worker jako kód psaný
    jazykem JavaScript a klientská knihovna pro interakci s ním. Komponenta
    podporuje použití Service Workers jako programovatelné HTTP proxy
    (\emph{fetch control}) a pro podporu upozornění generovaných serverem
    (\emph{push notifications}).

    Použití těchto knihoven pak demonstruji na třech aplikacích: TodoMVC, HNPWA,
    RealWorld, které jsou vytvořené podle specifikací běžně používaných pro
    porovnání webových platforem. TodoMVC je aplikace pro správu seznamu úkolů,
    HNPWA je klient zpravodajské služby Hacker News a RealWorld je
    víceuživatelská platforma pro čtení a psaní článků.

    Práce přidává do open-source ekosystému Haskellu tři nové komponenty, které
    v něm doposud chyběly. Tyto komponenty vyplňují základní nedostatky v
    ekosystému a dohromady funkčností odpovídají minimalistickým frameworkům v
    jazyce JavaScript, ale pořád v Haskellu zůstává mnoho nedostatků v porovnání
    s ekosystémem JavaScriptu. Tyto komponenty jsou již použitelné pro tvorbu
    Progressive Web Applications, zároveň ale slouží jako výchozí bod pro další
    práci, jejich další vývoj a tvorbu příbuzných komponent.

  },
}
