% Created 2019-02-01 Fri 16:37
% Intended LaTeX compiler: pdflatex
\documentclass[english,odsaz]{fitthesis}
\renewcommand\title[1]{}
%%% Local Variables:
%%% mode: latex
%%% TeX-master: "projekt.org"
%%% End:

\projectinfo{
  project={BP},
  year={2019},
  date=\today,
  title.cs={Název práce},
  title.en={Thesis title},
  %title.length={14.5cm},
  author.name={Jakub},
  author.surname={Zárybnický},
  department={UITS},
  faculty={FIT},
  supervisor.name={Ondřej},
  supervisor.surname={Lengál},
  supervisor.title.p={Ing.},
  supervisor.title.a={Ph.D.},
  keywords.cs={Sem budou zapsána jednotlivá klíčová slova v českém (slovenském) jazyce, oddělená čárkami.},
  keywords.en={Sem budou zapsána jednotlivá klíčová slova v anglickém jazyce, oddělená čárkami.},
  abstract.cs={Do tohoto odstavce bude zapsán výtah (abstrakt) práce v českém (slovenském) jazyce.},
  abstract.en={Do tohoto odstavce bude zapsán výtah (abstrakt) práce v anglickém jazyce.},
  declaration={Hereby I declare that this bachelor's thesis was prepared as an
    original author’s work under the supervision of Ing. Ondřej Lengál,
    Ph.D. All the relevant information sources used during preparation of this
    thesis are properly cited and included in the list of references.},
  %acknowledgment={Here it is possible to express thanks to the supervisor and
  %  to the people which provided professional help (external submitter, consultant, etc.).},
  extendedabstract={Do tohoto odstavce bude zapsán rozšířený abstrakt práce v
    českém jazyce, bude mít rozsah 2 až 6 normostran a bude obsahovat úvod,
    popis vlastního řešení a shrnutí a zhodnocení dosažených výsledků.},
}

\usepackage{minted}
\usepackage[figure,table]{totalcount}
\date{\today}
\title{}
\hypersetup{
 pdfauthor={},
 pdftitle={},
 pdfkeywords={},
 pdfsubject={},
 pdfcreator={Emacs 26.1 (Org mode 9.1.9)}, 
 pdflang={English}}
\begin{document}

% * (front matter)                                              :ignoreheading:
\maketitle
\setlength{\parskip}{0pt}
{\hypersetup{hidelinks}\tableofcontents}
\iftotalfigures\listoffigures\fi
\iftotaltables\listoftables\fi
\iftwoside\cleardoublepage\fi
\setlength{\parskip}{0.5\bigskipamount}

\chapter{Introduction}
\label{sec:org06f850b}
\begin{itemize}
\item a concrete motivating example of a Haskell app that needs a web client, plus
why to write a PWA
\item general trends in Haskell and frontend development
\item related work (Tweag, Obsidian, \ldots{})
\end{itemize}

\chapter{Technologies}
\label{sec:org94b3c11}
We'll first have a more detailed look at the above-mentioned technologies, [\ldots{}]

\section{Haskell}
\label{sec:org72a8337}
The programming language Haskell is a rather old one, despite spreading into the
industry only recently, having been around since 1990 [TODO: reference]. It is a
language that started out as a basis for research into the design of functional
languages. It has served as such, and in fact it still is the target of active
research - some more prominent projects are "Dependent Haskell" [TODO: ref] and
"Linear Haskell" [TODO: ref].

Haskell is described as a "statically typed, purely functional programming
language with type inference and lazy evaluation" [TODO: ref]. It is a language
in which its expressive type system enables very precise control over TODO:
expressive types, dialog with the compiler, types encoding effects, citation for
"If it compiles, it works".

TODO: language to write reliable software in - it eliminates entire classes of
programming errors - usually the ones that remain are logic errors.

While Haskell is not a language commonly associated with frontend development,
it is one of the many languages with the ability to use JavaScript as the
compilation target, instead of plain assembly or LLVM. In fact, such languages
have now become quite common in frontend development [TODO: ref], as is
exemplified by the rapid rise of TypeScript, a superset of ECMAScript 6 [TODO:
ref], or Elm, a framework with its own language based on Haskell [TODO: ref].

Of the many reasons for selecting a language other than JavaScript for frontend
development, one of the more notable ones is the ability to share code between
the server and its client in the case they are written in the same
language. This is the basic idea of the framework Meteor [TODO: ref], and in
fact the ability to run 'isomorphic code' - the same code on the client and the
server both - is its main marketing point.

TODO: Haskell on the frontend is a rapidly growing area with many interested
companies. The main projects in this area are sponsored and developed by them -
some of the main contributors are: Obsidian (Reflex, Obelisk), Tweag (Asterius,
Inline-js), and QFPL (several UI components). [TODO: refs] While it's a quickly
growing area, it's still not an established one, and there are many libraries
and tools still missing, ones that developers used to other languages have
learned to expect and require.

\begin{minted}[]{haskell}
type HackageAPI =
  "users" :> Get '[JSON] [User] :<|>
  "user" :> Capture "login" Login :> Get '[JSON] User :<|>
  "packages" :> Get '[JSON] [Package]

getUsers :: Handler [User]
getUser :: Login -> Handler User
getPackages :: Handler [Package]

server :: Server HackageApi
server = getUsers :<|> getUser :<|> getPackages

getUsers :<|> getUser :<|> getPackages =
  client @HackageApi "http://hackage.haskell.org"
\end{minted}

\section{Nix}
\label{sec:org2d5be15}
\begin{itemize}
\item \emph{"Nix package manager, a "purely functional" package and configuration
manager for computer systems"} -- Wikipedia
\item \emph{"Describe your end result, then magic happens"}
\item Tools build on top:
\begin{itemize}
\item Nix = package manager
\item NixOS = operating system
\item NixOps = cloud deployment tool
\end{itemize}
\item cross-compiling

\item since 2004, Eelco Dolstra's Ph.D. thesis in 2006
\item purely functional, lazy evaluation
\item one program consists of a closure that includes all dependencies including libc
\item atomic upgrades, rollbacks
\item complete isolation of dependencies, no DLL hell
\item NixOS = OS built on top of Nix
\item NixOps = a cloud deployment tool
\begin{itemize}
\item conceptually: Terraform/CloudFormation + Puppet/Ansible
\item a network specification -> magic -> running set of servers on AWS, VPSs,
VirtualBox, \ldots{}
\end{itemize}
\end{itemize}

\begin{minted}[]{nix}
{
  network.description = "Web server";

  webserver = { config, pkgs, ... }: {
    services.httpd.enable = true;
    services.httpd.adminAddr = "alice@example.org";
    services.httpd.documentRoot =
      "${pkgs.valgrind.doc}/share/doc/valgrind/html";
    networking.firewall.allowedTCPPorts = [ 80 ];

    deployment.targetEnv = "virtualbox";
  };
}
\end{minted}

\section{PWA}
\label{sec:org250af13}
\begin{itemize}
\item concrete examples of existing apps
\item general reasons for PWA development
\item Web Platform APIs, Service Workers
\end{itemize}

\chapter{Research}
\label{sec:org8d5aae7}
In order to write a SPA/PWA, there are some tools and supporting libraries that
a developer can't live without. We'll first walk through a high-level overview
of the common features of the Web frameworks of today, then go through the most
popular frameworks and see where they shine. Afterwards, we'll have a look at
the ecosystem of Haskell and try to find its strong and weak points, and
especially the places where the tools or libraries are entirely missing.

\section{Common features of Web frameworks}
\label{sec:org6d68e49}
-- A description of the things I'll be looking for, partially from
Wikipedia's \href{https://en.wikipedia.org/wiki/Comparison\_of\_web\_frameworks}{Comparison of web frameworks} and \href{https://en.wikipedia.org/wiki/Comparison\_of\_JavaScript\_frameworks}{Comparison of JavaScript
frameworks}, partially from my own experience, and partially from the feature
lists of the frameworks that I'll be looking into.

\subsection{Tooling}
\label{sec:org0909b17}
I'll start with the things you encounter first when setting up a project, its
tools. Developers have wildly differing levels of expectations from their tools. A
Python developer might expect just a text editor and an interpreter, whereas a
JVM developer might not be satisfied with anything less than a full-featured IDE.

\textbf{Code generators} or scaffolding tools start with creating a package manifest and
a \texttt{src/} directory, going on to generators that set up a few different types of
projects based on templates, all the way to tools that can add an entire website
module, perhaps even with database migrations.

\textbf{Build tools} range from a set of conventions on how to use your build tool that
might get formalized in your Makefile, through a CLI tool that takes care of
building, testing and perhaps even deploying your project, to the way of the IDE
where anything you can think of is just a few clicks away.

\textbf{Debugging tools} also come in many flavors and for many purposes. On the side of
the server, you have all the usual tools for the language, plus a few more - a
toolbar with an overview of everything that goes on in a page render or an AJAX
call, or the option to remotely connect to a running process and to debug
live. Client-side, we have the now irreplaceable DevTools with a built-in
debugger and profiler, but some frameworks go even further and provide a
framework-specific tools - React's component tree, or Elm's time-traveling
debugger.

\textbf{Quality assurance tools} have many sub-categories. From static code analysis
tools or linters, as they are commonly known; through tests - unit, integration,
end-to-end tests, or more exotic ones like marble tests or visual regression
tests; to profilers - runtime or allocation measurements, frontend performance
measurements, or more involved tools like performance evolution tracking.

\subsection{Features}
\label{sec:org71e19d2}
TODO: Reread and clean up

\textbf{Templating}, which at the frontend means a way to compose the HTML that makes up
an application, usually including some render logic and variable
interpolation. In some frameworks the whole program is a template (see React),
some have templates in separate files and pre-compile them during runtime (see
Angular). Templates sometimes contain CSS as well (see the new CSS-in-JS trend)

\textbf{Forms} are the basic element of interactive applications. There are a few layers
of abstraction at which a framework can decide to implement forms - starting at
raw DOM manipulation, going on to data containers with validation (but manual
rendering), all the way up to form builders, manual or automatic. Under 'forms'
I count a way to render a form, to validate user input, and collect the result.

\textbf{Routing} means manipulating the displayed URL using the History API and changing
the application state to reflect it, and also loading the correct state on
application start-up. This can also include animated transitions between pages.

\textbf{Internationalization} has many levels. At its most basic, it means simple string
translations, pluralization, and word order. Going further, it means also RTL
scripts, date/time formats, currency, or time zones.

Modern web frameworks also provide a way to use a web application's code and
compile it into a \textbf{native mobile application}. It's commonly implemented as a
wrapper application around web apps using Cordova, but technologies like React
Native go further and use native UI elements. Mobile applications can be faster
to use and faster to load, can access device-specific APIs not exposed via Web
Platform APIs.

Some frameworks also support creating a \textbf{desktop application} that reuses a web
application's code. The underlying technology here is Electron and not Cordova,
but the benefits are the same - greater speed and access to device-specific
functionality.

\subsection{Accessibility (ARIA, \ldots{})}
\label{sec:orgc1fdb11}
\begin{itemize}
\item the key word now is ARIA = support for screen readers
\item also, semantic elements, text contrast, customization
\item also also, keyboard-accessibility (shortcuts, every clickable element
accessible via keyboard = tabindex)
\item accessibility testing (automatic as well, see aXe)
\end{itemize}

\subsection{Optimistic updates}
\label{sec:org22f479c}
\begin{itemize}
\item one of the things I want to focus on
\item broadly, expecting that every network request will be successful and updating
the GUI accordingly
\item rolling back app state in case of failure, with notifications
\end{itemize}

\subsection{Web Platform}
\label{sec:org8a17782}
\begin{itemize}
\item location, camera, touch, vibration, \ldots{}
\end{itemize}

\subsection{Pre-render}
\label{sec:org25dd468}
\begin{itemize}
\item one approach to shortening start-up times
\item serving HTML with all the content already inside, no need for more requests to
the backend for the initial page load
\item JS takes over and uses what's already been loaded
\item can be static or dynamic:
\begin{itemize}
\item static = JAM stack, serving a bunch of files rendered at compile-time
\item dynamic = rendering the HTML at runtime
\end{itemize}
\end{itemize}

TODO: What about the backend feature write-ups I have?

\section{JavaScript ecosystem}
\label{sec:org17dc5f4}
\subsection{Angular}
\label{sec:org7f55e9e}
TODO: what is it?
On a first look, Angular looks like a well thought-out frontend
framework. Written in Typescript with comprehensive documentation and great
tooling, it seems that the authors have learned from their mistakes with
AngularJS.

Some notable features:
\begin{itemize}
\item command line tool, \texttt{ng} - it streamlines setting up the entire project -
scaffolding, preparing build and testing tools, starting a
development server, \ldots{}
\item runtime environments - from server-side rendering, PWAs with ServiceWorkers,
to native and desktop applications, it seems that Angular tries to cover every
possible use-case
\item tooling other than the \texttt{ng} tool - browser extensions for runtime debugging,
IDEs and others. I haven't thought of a tool I would miss, but I'm used to
minimalism in tooling from the Haskell world\ldots{}
\end{itemize}

Some negatives that developers complain about:
\begin{itemize}
\item Angular is intimidating for a new developer, it's too complex and there's a
lot to learn
\item Too much 'magic' - related to the previous point, there's a lot of abstraction
and it's not easy to understand all the layers
\item Code bloat - the amount of boilerplate and also the size of the resulting bundle
\item Too opinionated - if you don't like 'the angular way', you're out of luck here
\item scattered documentation - too many articles and tutorials out there for
AngularJS that can't work with the new Angular
\end{itemize}

\subsection{React}
\label{sec:org6677fa9}
React is not a framework in itself. Rather, it's a library that focuses on a
single thing and does it in a unique enough way that there's sprung up an entire
ecosystem around it. In it, there are groups of libraries that build upon React,
each focusing on a single feature - UI components, state management, forms etc.

There's a large jungle of libraries, each one with a different scope and
focus. Choosing a library that fits your problem can sometimes take many
attempts. Add to it the fact that libraries, frameworks and tools come and go
quite quickly - the main cause of the so-called "JavaScript fatigue - and the
fact that in JavaScript, it's fashionable to write extremely small libraries,
and you have a recipe for a quite unpleasant development experience.

I'll try to go through some of the most popular 'frameworks' that build on
React, though each one is more of a pre-built toolkit of libraries and tools
and bits of glue in between, rather than cohesive frameworks. In general, the
React world is a lot more mix-and-match than developers used to enterprise
frameworks would expect.

Create-react-app, nwb, Razzle, and Neutrino all cover only the build
process. Next.js is the first one that I've found that goes a step beyond just
pre-configuring Webpack and other build tools - it provides other features that
are starting become standard - server-side runtime rendering, link prefetching,
and build-time prerendering. It's also the first tool I found that considers
that a website can consist of multiple applications, via its 'zones' feature.

TODO: developers' opinions

\subsection{Gatsby}
\label{sec:orgd101388}
One rather unique framework I found - and this is a framework in a strong
sense, not like the React tools above - is Gatsby. It's unique in the sense that
while it's a frontend framework, it's not supposed to run in a browser. It's a
part of a growing movement centered around the 'JAM stack' - "JavaScript, APIs,
and Markup". That doesn't tell you much, but the main feature is that at
build-time, you fetch data from your APIs, and render the application to plain
HTML files, so that you don't need a server other than an S3 bucket or similar.

It's a framework targeted at a specific subset of website - not single-page
applications, but more blogs or e-shops, and a workflow exemplified by Netlify.
This means it doesn't need to concern itself with many features that would be
missing in a frontend framework intended for a browser, and those are delegated
to a different part of the stack.

TODO: opinions, structured pro/cons

\subsection{Vue.js}
\label{sec:org317ae1d}
TODO: Vue.js

\subsection{Polymer}
\label{sec:org9edf765}
TODO: Polymer

\section{PHP ecosystem}
\label{sec:org20cd722}
TODO: PHP ecosystem

\section{Haskell ecosystem}
\label{sec:orge80f0df}
TODO: Compare Haskell compared to the above list

\begin{itemize}
\item its strong points
\item passable but not ideal libraries
\item what's missing
\end{itemize}

\subsection{UI toolkit}
\label{sec:orgfb5f79d}
At the frontend, the UI toolkit defines what the entire application will look
like, its architecture as well as the tools it can use. There is significant
pressure on well-defined types in Haskell, and the UI libraries have pushed
strongly in favor of FRP (Functional Reactive Programming) or its derivatives
(see 'Elm architecture' [TODO: ref]).

I've managed to find five production-ready libraries that are usable with
GHCJS. Out of these five, \texttt{react-flux} and \texttt{transient} are unmaintained, and \texttt{reflex},
\texttt{miso}, and \texttt{concur} all wildly differ in their philosophy, architecture, and
maturity.

\texttt{Reflex} ([TODO: ref] and \texttt{reflex-dom} [TODO: ref], its DOM bindings) seems like the
most actively maintained one, and also the most promising one regarding its
future outlooks - it's sponsored by Obsidian Systems [TODO: ref, again] and
actively developed for general use on GitHub. [TODO: describe its FRP, \ldots{}]

\texttt{Miso} [TODO: ref] - TODO: Elm architecture re-implementation in Haskell

\texttt{Concur} [TODO: ref] - TODO: an experimental architecture but actively developed,
variants in PureScript as well. A technology to investigate in the future, but
no ecosystem right now and not fully mature.

TODO: examples of Reflex, Miso, Concur

\subsection{Build tools}
\label{sec:org4af2283}
The UI toolkit constrains the choice of possible build tools. In Haskell, there
are three mature options - Cabal (new-build) [TODO: ref], Stack, and Nix. There
is also a new fourth option aiming for the best of these four, Snack [TODO:
ref]. While it's not yet mature enough for serious use, it's a tool worth future
investigation.

Cabal - old, Cabal hell. Stack came, divided the Haskell community. Nix came out
of nowhere, converted a significant portion but isn't yet dominant, and now
cabal new-build is almost equivalent to stack. Free choice between cabal and
stack, nix is more capable in general but slower learning curve. TODO: more
about snack

GHCJS ecosystem not so well supported with Cabal (old or new one), Nix is
usually recommended at the frontend (one command setup, binary cache,
cross-platform), Reflex especially - uses the great Nix cross-compilation
capabilities for Android, iOS, desktop. Stack usable for plain GHCJS dev, but
old GHC (7.10?).

Slow compiler - common workaround is to use the REPL, but there are other
solutions like Snack which are promising although not widely used.

GHCJS output size and speed - GHCJS should be obsoleted by WebAssembly very
soon - it's already in alpha state, and I expect it will be ready for
production by the end of 2019.

Lack of editor integration - solved with LSP+HIE (usable in VSCode and other modern
editors), but Emacs is still the safest choice. Hlint (linter), Hindent
(formatter) built into HIE and Intero both, same goes for 'go-to-definition' and
'type-at-point' features common to modern IDEs. What's missing is debugger
integration - usually via GHCi only, but projects like haskell-dap (+ Phoityne
editor plugin) exist.

\subsection{UI components}
\label{sec:orge5f0ab9}
Preferred approach - implement UI and logic inside application code. FFI is
usable and quite simple for bigger components (see reflex-dom-ace).

Reflex and Concur - self-contained components and global state both (-> reusability)
Miso - single state atom only (see TEA)

\subsection{Forms}
\label{sec:org8d8e902}
No good options exist for any of the frameworks. While there are some attempts
at a forms library for Reflex, there is not a single feature-complete
library. Part of the reason - validation-only libraries exist in Haskell and
commonly used (see \texttt{validation}, \texttt{digestive-functors}). There are forms libraries at
the backend (see \texttt{yesod-form}) with almost automatic form generation.

\subsection{Routing}
\label{sec:org8dc81d7}
Miso has routing built-in. There are several attempts at a Reflex routing
library but not a single accepted solution. Concur with its small ecosystem
doesn't have even that.

\subsection{Mobile/desktop apps}
\label{sec:orgc8dbab7}
Reflex has this built-in via reflex-platform. Obelisk, building on top of
reflex-platform, includes bundling apps for App/Play Store. Cross-compilation via
Nix to Android/iOS, reflex-dom's bindings to WkWebView on desktop. I haven't
found any attempt to do this for Miso or Concur (Concur has beginnings of React
Native and SDL backends, but the development seems to have stalled).

\subsection{Accessibility}
\label{sec:orgbc6e532}
I haven't found anything related to this topic, so I must assume no one has even
attempted to tackle this topic. Building accessible applications by yourself
isn't hard though.

\subsection{Optimistic updates}
\label{sec:orgabf7642}
I haven't found anything related to this topic. For FRP though, this should be
implemented at the data or network layers though, so this is something to work
towards when building the 'offline storage' library as per my assignment.

\subsection{Web Plaftorm}
\label{sec:org4720514}
The library JSaddle wraps the entirety of the Web Platform APIs using
WebIDL. Any UI library can use this library, though there are limited
event-based bindings, which means writing manual wrappers. (Reflex has some
machinery for a subset for it, Miso has subscriptions for some of them, Concur
uses ghcjs-vdom which has also some limited bindings.)

\subsection{Prerender (Isomorphic rendering)}
\label{sec:org3690e0d}
Reflex has explicit support, \texttt{preRender} allows even for two variants of an
element if it doesn't render well. Miso has explicit support as well. I haven't
found anything like this for Concur.

\subsection{Support tools}
\label{sec:orgfe6a32e}
Linter - Haskell standard is Hlint, support for custom rules, well-supported and
mature. Code formatter situation is more divided, Hindent follows gofmt with a
single code style for all code (but doesn't cover edge cases esp. for type-heavy
code, so not ideal). Brittany is an ambitious project using GHC's parser itself,
nicely designed formatting rules, but supports only a limited subset of the
language. Several other projects, some unmaintained, some brand new, but Hindent
seems to be standard at the moment.

The situation around code generators isn't ideal. Stack bundles several
templates, but is limited to Stack users. Cabal has a single template, nothing
else. Nix doesn't care about scaffolding. Several other tools, mostly
unmaintained - Summoner is a notable one, with a fixed project structure; Hi is
another one, supports arbitrary templates. There is no standard tool, usually
only 'git clone' a project template and start developing. Obelisk has a standard
structure, but it's a very limiting one (Snap as the only server library,
predetermined routing library obelisk-route, \ldots{}).

TODO: more code generation tooling - 'rails new
controller/scaffold/module/model/migration'

\subsection{Quality Assurance}
\label{sec:org88b1d3c}
On this front, Haskell is very developed, at least theoretically. QuickCheck
originated in Haskell and quickly propagated to many other languages - followup
tools like SmallCheck and similar. That's generative, or property-based testing,
best in class.

Classical unit and integration testing has also many options - hspec, hedgehog,
tasty, doctest, \ldots{} Mocking via free monads or other MonadX
implementations. TODO

End-to-end testing - Selenium webdriver wrappers exist, and Selenium is the
standard for testing servers and clients. (Or alternately shelltestrunner for
testing CLI based applications.)

Benchmarking - best in class - criterion, no competition. Weigh - allocation
measurements as a benchmark.

\subsection{Documentation}
\label{sec:org5feaa75}
It is generally agreed that documentation is Haskell's weakest point. Despite
having a great standard API docs tool (haddock), documentation is often an
afterthought, with incomplete docs or unclear starting points (which means no
tutorials etc. either).

\subsection{Server-side tools}
\label{sec:orgdaa3c09}
\begin{itemize}
\item Communication - typed APIs (HTTP - servant, WebSocket - some attempts, nothing
production-ready)
\item Entities - many options, some great, some less
\item Migrations - weak point, I've found many half-baked implementations, but no
standard solution (not only a Haskell problem, Liquibase is the only standard
here, and that's SQL only)
\end{itemize}

\section{Implementation plan}
\label{sec:org402967f}
TODO: Somehow use the "Evolving frameworks" article

In this thesis, I'll focus mainly on the tools that are necessary specifically
for PWA development, not general quality-of-life libraries.

\textbf{Goal = app that fulfills the basic PWA criteria}:
\begin{itemize}
\item Pages are responsive on tablets \& mobile devices
\item All app URLs load while offline
\item Metadata provided for Add to Home screen
\item Page transitions don't feel like they block on the network
\item Each page has a URL
\item Pages use the History API
\item Site uses cache-first networking
\item Site appropriately informs the user when they're offline
\item Push notifications (consists of several related requirements)
\end{itemize}

To achieve this, I need to create:
\begin{itemize}
\item A full-featured browser routing library. While there are some existing
implementations, they are either incomplete or long abandoned.
\item A wrapper around ServiceWorkers, or a template to simplify project creation.
\item A push notifications library.
\item A library or a script that will render HTML 'shells' of all pages on the site,
for fast first load.
\item CLI tool (requirement from assignment? "support tools")
\item An offline storage library (some variant of it)
\end{itemize}

Not required by the checklist, but would improve the quality of my work:
\begin{itemize}
\item A template of an application, with predefined internal architecture, that uses
all of the above libraries
\item A utility library for querying and caching data from an API, be it HTTP, WebSocket,
GraphQL, or others.
\end{itemize}

\textbf{Stretch goals}:
\begin{itemize}
\item a library to use in code shared between the server and client - a way to
define the shape of the transport channel (and its API for non-Haskell
applications)
\item a server library, to allow user code to implement the specified protocol
\item a client library with a storage component for entities and pending requests
\end{itemize}

TODO: talk about how these tools will fit into a 'platform', so that I check of
a box from the assignment

An application using these tools and libraries would consist of a server written
independently, and of a browser application. Implementing the communication
between them is left to the developer, as is implementing many common
conveniences usually provided by a framework.

\textbf{Stretch}: An application using these libraries would consist of a server and a client
sharing code that contains the definition of their communication channel. I do
not yet know how much these libraries would affect the shape/architecture of the
server, but the client library would form the core of the client - with the
libraries from the previous plan forming the shell.

These goals also include fully documenting the code written, as well as testing
and benchmarking it to remove the most obvious bottlenecks.

TODO: talk about Serverless, JAM stack, and alternative app structures:
\begin{itemize}
\item Client only. An application that doesn't need to communicate with a server,
like a web presentation, or a blog, a set of pre-compiled HTML+JS files.
\item Server only. Either just an API, or a plain HTML website with no JavaScript.
\item Server and client, with the client rendered during run-time by the server.
\item Server and client, with the client rendered during build-time and served
separately (e.g. via an S3 bucket).
\item Server and client, with the client re-rendered on demand, whenever the data that
it shows changes. This is the shape of a 'JAM stack' application.
\item Projects with multiple clients and/or servers (even more pressure on
supporting tools)
\end{itemize}

\chapter{Components}
\label{sec:org8a5b091}
TODO: Demonstrate the principles of components on 'src-snippets' code, where
I'll show the smallest possible code that implements that functionality

\section{Component A}
\label{sec:orgb786ab8}
\subsection{Design}
\label{sec:orge19c543}
\subsection{Implementation}
\label{sec:org253c664}
\subsection{Testing}
\label{sec:org1f810eb}
\subsection{Other options, possible improvements}
\label{sec:org5b8ffb9}

\chapter{Applications}
\label{sec:org2418a9d}
\section{Workflow and tools}
\label{sec:org5838504}
TODO: describe the development flow of an app built using these tools

\begin{itemize}
\item starting out - three layer cake \& esp. the inner one
\item QA (tests, e2e, CI, \ldots{}), documentation
\item development tool options
\item deployment options
\end{itemize}

\section{TodoMVC}
\label{sec:org9039ca9}

\section{RealWorld}
\label{sec:org33da4c1}

\chapter{Conclusion}
\label{sec:org3eef492}
TODO: return to the comparison with JS, PHP, \ldots{} frameworks

TODO: describe possible follow-up work, what I'll be working on - define
  specific topics and make concrete examples

The final chapter includes an evaluation of the achieved results with a special
emphasis on the student's own contribution. A compulsory assessment of the
project's development will also be required, the student will present ideas
based on the experience with the project and will also show the connections to
the just completed projects. \cite{Pravidla}

% * (bibliography, start of appendix)                           :ignoreheading:
\makeatletter
\def\@openbib@code{\addcontentsline{toc}{chapter}{Bibliography}}
\makeatother
\bibliographystyle{bib-styles/englishiso}

\begin{flushleft}
\bibliography{projekt}
\end{flushleft}
\iftwoside\cleardoublepage\fi

% Appendices
\appendix
\appendixpage
\iftwoside\cleardoublepage\fi

\startcontents[chapters]
% \setlength{\parskip}{0pt}
% \printcontents[chapters]{l}{0}{\setcounter{tocdepth}{2}}
% \setlength{\parskip}{0.5\bigskipamount}
\iftwoside\cleardoublepage\fi

\chapter{Contents of the attached data storage}
\label{sec:org43411b8}
TODO: fill in

\chapter{Poster}
\label{sec:org8c111fa}
TODO: fill in
\end{document}