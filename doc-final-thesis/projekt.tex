% Created 2018-09-04 Tue 20:24
\documentclass[english,odsaz]{fitthesis}
\renewcommand\title[1]{}
%%% Local Variables:
%%% mode: latex
%%% TeX-master: "projekt.org"
%%% End:

\projectinfo{
  project={BP},
  year={2019},
  date=\today,
  title.cs={Název práce},
  title.en={Thesis title},
  %title.length={14.5cm},
  author.name={Jakub},
  author.surname={Zárybnický},
  department={UITS},
  faculty={FIT},
  supervisor.name={Ondřej},
  supervisor.surname={Lengál},
  supervisor.title.p={Ing.},
  supervisor.title.a={Ph.D.},
  keywords.cs={Sem budou zapsána jednotlivá klíčová slova v českém (slovenském) jazyce, oddělená čárkami.},
  keywords.en={Sem budou zapsána jednotlivá klíčová slova v anglickém jazyce, oddělená čárkami.},
  abstract.cs={Do tohoto odstavce bude zapsán výtah (abstrakt) práce v českém (slovenském) jazyce.},
  abstract.en={Do tohoto odstavce bude zapsán výtah (abstrakt) práce v anglickém jazyce.},
  declaration={Hereby I declare that this bachelor's thesis was prepared as an
    original author’s work under the supervision of Ing. Ondřej Lengál,
    Ph.D. All the relevant information sources used during preparation of this
    thesis are properly cited and included in the list of references.},
  %acknowledgment={Here it is possible to express thanks to the supervisor and
  %  to the people which provided professional help (external submitter, consultant, etc.).},
  extendedabstract={Do tohoto odstavce bude zapsán rozšířený abstrakt práce v
    českém jazyce, bude mít rozsah 2 až 6 normostran a bude obsahovat úvod,
    popis vlastního řešení a shrnutí a zhodnocení dosažených výsledků.},
}

\usepackage[figure,table]{totalcount}
\date{\today}
\title{projekt}
\begin{document}

% * (front matter)                                              :ignoreheading:
\maketitle
\setlength{\parskip}{0pt}
{\hypersetup{hidelinks}\tableofcontents}
\iftotalfigures\listoffigures\fi
\iftotaltables\listoftables\fi
\iftwoside\cleardoublepage\fi
\setlength{\parskip}{0.5\bigskipamount}

\chapter{Introduction}
\label{sec-1}
In order to be able to write a professional text clearly and comprehensibly, we
have to meet several basic prerequisites:
\begin{itemize}
\item we need to have something to say,
\item we need to know who we want to target with the text,
\item we have to thoroughly think about the content,
\item we have to write in a structured way.
\end{itemize}

These and other instructions are also available on the school's website
\cite{fitWeb}.

An overview of the basics of typography and document creation using the system
\LaTeX{} is shown in \cite{Rybicka}.

\section{We need to have something to say}
\label{sec-1-1}
Another important prerequisite for good thesis is to \emph{write for someone}. If
we write notes for ourselves, we write them differently than a research report,
an article, a diploma thesis, a book or a letter. According to the assumed
reader, we decide on the type of writing, the scope of information and the
degree of detail.

\section{We need to know who we want to target with the text}
\label{sec-1-2}
Another important prerequisite for good thesis is to \emph{write for someone}. If
we write notes for ourselves, we write them differently than a research report,
an article, a diploma thesis, a book or a letter. According to the assumed
reader, we decide on the type of writing, the scope of information and the
degree of detail.

\section{We have to thoroughly think about the content}
\label{sec-1-3}
We need to thoroughly think about the content and compose the content of the
message and create the order in which we want to present our ideas to the
readers. Once we know what we want to say and to whom, we need to plan out the
subject matter. Ideal is such a layout, which makes a logically accurate and
psychologically digestible whole, in which everything is in place and whose
parts fit into each other. All contexts are clear and it is obvious where they
belong.

In order to achieve this, we must thoroughly organize the subject matter. We
will decide what the main chapters and subchapters would be and what are the
connections between them. The chart of such an establishment is a chart that is
very similar to a tree, but not a string. When organizing a subject matter, the
question of what to include in the schema is equally important as a question of
what to not include. Too many details can be discouraged by readers as well as
no details.

The outcome of this stage is the outline of the text, which is the sequence of
the main ideas and the details included therein.

\section{We have to write in a structured way}
\label{sec-1-4}
We must start writing in a structured way, and at the same time we need to use
the most comprehensible form, including good style and perfect marking.  As soon
as we have an idea, an idea of a future reader, a goal, and an outline of the
text, we can start writing. When writing a first concept, we try to capture all
of our thoughts and opinions regarding to individual chapters and
subchapters. Each thought must be explained, described and proven. The main
sentence should always express the main idea, not the secondary.

We should approach the writing process in a structured way as well. At the same
time, as we clarify the structure of the written work, we create key points of
text that we gradually update. We use those DTP tools that support structured
text construction (predefined types for titles and text blocks).

\chapter{Several formal rules}
\label{sec-2}
Our goal is to create a clear and comprehensible text. Therefore, we express
ourselves properly, we use good Czech (or preferably English) and a good style
according to the generally accepted customs. The text should provide readers
with a way to quickly understand the problem, anticipate its difficulties, and
prevent them. A good style requires impeccable grammar, correct punctuation, and
appropriate word choice. Avoid using a small selection of words otherwise the
text might become monotonous, also avoid using some of your favorite words too
often. If we use foreign words, it is assumed that we know their exact
meaning. We also must use Czech words in the right sense. It is not wrong to use
the author's \emph{we}, so we assume that we are solving something, or for
example, generalizing with the reader. In bachelor or master thesis author's
\emph{I} can be used (for example, when we define a share of our own work in
relation to the translated text), but in casual text excessive use of the first
person of the single number is not recommended.

Make sure to precisely select symbols that we use to \emph{mark}. We mean the
choice of abbreviations and symbols used, for example, to express the types of
components to identify the main program functions, to name lethal keyboard keys,
to name variables in mathematical formulas, and so on. Appropriate and
consistent labeling can greatly benefit reading experience for readers. It is
appropriate to give a list of the markings at the beginning of the text. Not
only in marking, but also in references and in overall layout, consistency is
important.

This is also associated with a typology term called \emph{highlighting}. Here we mean
the typesetting for its highlighting. The selected method of marking should
match the selected highlighting. For example, the keyboard keys can be placed in
a rectangle, the identifiers from the source text may be written in a
\verb~typewriter font~, and so on.

If you present some facts, do not conceal their origin and your relationship to
them. When you claim something, always explicitly state what has been proven and
what will be proven in our text and what we took over from literature with
reference to the source. In this respect, never leave the reader in doubt
whether this is our idea or an idea taken from literature.

Never waste reader's time with the interpretation of trivial and insubstantial
information. Also do not say the exact thing just in different words. With later
modifications to the text, some previously written passage may seem
unnecessarily detailed or even totally useless. Dropping such passage or at
least making it briefer will contribute to better readability of the work! But
this step requires the courage to throw away the time we have devoted to
creating it.

\chapter{It will never be perfect}
\label{sec-3}
When you have already written everything you have been thinking about, take a
day or two days off, and then read the handwriting again. Make your last edits
and finish. We are aware that there is always something left unfinished, there
is always a better way of explaining something, but each stage of the adjustment
must be final.

\chapter{Typographic and linguistic principles}
\label{sec-4}
When printing \emph{technical report}, When typing a technical text type, a
technical report, such as the text of the qualification work, A4 format is often
chosen and we often print only on one side of the paper. VIn that case, make the
left margin of all pages to be slightly larger than the right - at this place
the papers will be bound and the binding technique will force this
requirement. When bounding with a rigid back, the left edge should be slightly
wider for thick bundles because the pages will be harder to open and the left
margin will be less exposed to the eye.

Select the upper and lower edges the same size, or move the printed part
slightly upwards (the upper edge is smaller than the lower edge). Keep in mind
that the edges will be slightly cropped after binding.

For an A4 page, it is appropriate to use a font of 11 points for basic
text. Choose a width of 15 to 16 centimeters and a height of 22 to 23
centimeters (possible headers and footers included). Line spacing usually should
be 120 percent of the font used, which is the optimal value for the reading
speed of the contiguous text. If you decide to use \LaTeX{}, use the default
settings. When writing a qualification work, make sure to follow the mandatory
requirements.

The font level for different levels of title is selected according to standard
typographic rules.  Typically, for all types of headings semi-bold or bold font
are being used (uniformly either semi-bold everywhere or bold everywhere). Size
of line spacing is chosen that the following text of regular paragraphs is
preferably set on a \emph{fixed index}, that is to say on lines with a predefined
and fixed spacing.

The arrangement of the individual parts of the text must be clear and
logical. It is necessary to distinguish the names of the chapters and
subchapters --write them in lowercase letters except for the capital starting
letters. For each paragraph of the text, offset the first line of the paragraph
with about one to two squares (always the same preselected value), thus about
two widths of the capital letter M of the basic text. The last line of the
preceding paragraph and the first line of the following paragraph are not
separated by a vertical gap in this case. Spacing between these lines is the
same as the spacing between the lines inside the paragraph.

When adding images, choose their size so that they do not exceed the area onto
where the text is printed (thus text edges from all sides). For large images,
use a separate page. Place pictures or spreadsheets of sizes larger than A4 in a
written message in the form of a booklet embedded in an attachment or embedded
in the tabs on the backboard.

Pictures and tables must have sequential numbering. The numbering is chosen
either continuous throughout the text, or -- which is more practical --
continuous within the chapter. In the second case, the table or image number
consists of the chapter number and the number of the picture / table in the
chapter - the numbers are separated by a dot. The numbers of subchapters have no
effect on the numbering of pictures and tables.

Tables and pictures use their own, independent numerical series. Thus in the
references inside the text we must also provide information about whether we
refer to a picture or a table (for example ``\ldots \emph{see table
2.7}\ldots``). Additionally, following this principle is very natural.

For sitelinks, chapters and subchapters, figure numbers and tables and for other
similar examples, we use special DTP programs to ensure that the correct number
is generated even if the text is moved by changes in the text itself or by
adjusting the style parameters. An example of such a too in \LaTeX{} is a reference
to the corresponding location of the tag in the text, such as a label
(\verb~\\ref\{navesti\}~ -- according to the location of the labels it will be the
number of the chapter, subchapter, picture, table, or similar numbered element),
the page that contains the tag (\verb~\\pageref\{navesti\}~), or a literary reference
(\verb~\\cite\{identifikator\}~).

The equation to which we refer in the text is given serial numbers at the right
margin of the corresponding row. These sequence numbers are written in
parentheses. The equation numbering can be continuous in the text or in
individual chapters.

If you are in doubt when typesetting a mathematical text, try to keep the \LaTeX{}
defined typesetting. If your work contains a large number of mathematical
formulas, we recommend using the \LaTeX{} system.

Do not make a space where digits are combined with letters in one word or one
character.  Punctuation such as dot, comma, semicolon, colon, question mark and
exclamation mark, as well as closing brackets and quotation marks are appended
to the preceding word without a space. The space is behind them. However, this
does not apply to decimal points (or decimal dots). The opening bracket and the
front quotes are appended to the following word and the space is omitted before
them -- (like this) and\textasciitilde{} \verb~like this~.

We do not use the same character for hyphen and dash. For a dash another
character is reserved (longer). In the \TeX{} system (\LaTeX{}), the hyphen is written
as one character \verb~hyphen~ (example \verb~Brno-město~), For an interval or
pairs, rivals, and similarly the source text uses a pair of characters \verb~dash~
(such as \verb~match Sparta -- Slavie~; \verb~price 23--25 Crowns~), For the
distinctive separation of a part of the sentence, for the distinct separation of
the inserted sentence, for the expression of an unspoken idea, and in other
situations (see Czech Spelling Rules), the longest type of dash is used, which
is written in the source text as three characters \verb~dash~ (such as \textasciitilde{}Another
term --- however it may seem insignificant --- will be informally defined in the
following paragraph.\textasciitilde{}). For the mathematical minus symbol, a different character
is used. In \TeX{} system it is written as a normal minus symbol in the source text
(thus symbol \verb~dash~). The typesetting in the mathematical environment where
the formula is surrounded by dollars will ensure that the correct output is
generated.

The forward slash is written without spaces. For example, the school year
2008/2009.

The rules for writing abbreviations are set out in Czech Spelling Rules
\cite{Pravidla}. For other reasons, it is appropriate to have this book at hand.

\section{What is a standard page?}
\label{sec-4-1}
Term \emph{standard page} refers to the assessment of the extent of the work, not
the number of sheets printed. From the historical point of view, it is the
number of pages of manuscript written on a typewriter on special preprinted
forms, with an average length of 60 characters per line and 30 lines per page of
the manuscript. Because of the correctio nmarkers, line spacing 2 (every other
row) was used. These data (the number of characters per line, the number of rows
and the line spacing between them) do not determine to the final printed
result. They are only used for range assessment. One standard page is therefore
60 * 30 = 1800 characters. Images included in the text are counted in the scope
of the written work approximately same as the amount of text that would produce
the same size in the resulting document.

The approximate range of work in standard pages can be determined by the \emph{Word
Count} function in the Microsoft Word \emph{Tools menu} by dividing the value
\emph{Characters (including spaces)} by constant 1800. Only the text written in the
core of the work is included in the scope of work. Parts such as abstract,
keywords, statements, content, literature, or attachments do not count towards
the scope of work. Therefore, it is necessary first to select the core of the
work and then have the number of characters counted for you. You can estimate
the approximate range of images manually. Similarly, you can use
OpenOffice. When using \LaTeX{}, the situation is a bit more complicated. For a
rough estimate of the number of standard pages, you can use the sum of sizes of
the source files of the work divided by a constant of about 2000 (normally we
would divide by 1800, but in the source files there are also commands which are
not counted in the range). For a more accurate estimate, plain text from PDF can
be extracted (for example, using the cut-and-paste or \emph{Save as Text\ldots{}} method
and divide it by 1800.

\chapter{Conclusion}
\label{sec-5}
The final chapter includes an evaluation of the achieved results with a special
emphasis on the student's own contribution. A compulsory assessment of the
project's development will also be required, the student will present ideas
based on the experience with the project and will also show the connections to
the just completed projects.
% * (bibliography, start of appendix)                           :ignoreheading:
\makeatletter
\def\@openbib@code{\addcontentsline{toc}{chapter}{Bibliography}}
\makeatother
\bibliographystyle{bib-styles/englishiso}

\begin{flushleft}
\bibliography{projekt}
\end{flushleft}
\iftwoside\cleardoublepage\fi

% Appendices
\appendix
\appendixpage
\iftwoside\cleardoublepage\fi

\startcontents[chapters]
% \setlength{\parskip}{0pt}
% \printcontents[chapters]{l}{0}{\setcounter{tocdepth}{2}}
% \setlength{\parskip}{0.5\bigskipamount}
\iftwoside\cleardoublepage\fi

\chapter{Obsah přiloženého paměťového média}
\label{sec-6}
\ldots{}
\chapter{Plakát}
\label{sec-7}
\ldots{}
% Emacs 25.3.1 (Org mode 8.2.10)
\end{document}